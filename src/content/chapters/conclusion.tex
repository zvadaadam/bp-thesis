\chapter{Conclusion}

The goal of the thesis was to get familiar with the speech recognition field and implement speech recognizer using neural networks which would be used as voice-user interface on Robot NAO.

We have covered the topics of artificial neural networks and recurrent neural network, and we explained how backpropagation algorithm works during a training phase.
Afterwards we explored speech recognition architectures and explained how speech signals is modified for the purposes of speech recognition.

The implemented solution of end-to-end speech recognizer is build upon recurrent neural networks with LSTM neurons, CTC loss function and speech signal features are extracted using MFCC.
The recognizer was firstly trained on Free Spoken Digit Dataset where we achieved error rate of 3\% but the model was overfitted.
We have tried to tweak the hyperparameters for better performance and use dropout as optimization technique against overfitted.
We have successfully lowered the error rate on 2\%.
The main end-to-end speech recognizer which will be used in Robot NAO, used VCTK Corpus as training dataset and achieved error rate of $3\%$.

In future work we want to finish the integration of implemented speech recognition with Robot NAO and train the RNN on more complex speech corpus with deeper network.
We would like to improve the recognizer by using bidirectional recurrent neural networks and trained on large dataset with properly prepared validation and test data.

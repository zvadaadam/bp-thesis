\chapter{Conclusion}

The goal of the thesis was to implement speech recognizer using nerual networks which will be later used as voice-user interface on Robot NAO.

First we have introduced the concept of artifical neural networks and basics of the internal network architecture.
We have explained the training phase of artifical neural networks and how backpropagtion algortihm is used to modify the weights and baises.
We have extanded the knowleadge of nerual networks by introducing recurrent nerual networks and explained the improved version of backpropagtion algortihm, compatible with RNN.
However, RNNs presented a problem with vanishing and exploding gradient and we explained a possible solution in type of LSTM cells as an improvment upon basic RNN's neuron.

The speech recognition systems has drasticly changed over the last decade.
We have given an overview of HMM-based speech recognition system and talked about how the field is starting to shift to end-to-end speech recognizer systems.

We have introduced the speech recognition system based on reccurent nerual networks with LSTM cells and CTC loss function on the output layer and propoused the implementation of such a system using TensorFlow library.
Validation of such a model was first performed on audio number dataset where

In future work we want to finish the integration of implemented speech recognition with Robot NAO and most imporantly traine it on more advance speech corpus with deeper network.

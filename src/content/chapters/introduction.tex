\chapter{Introduction}

The problem of speech recognition (SR) has been an important research topic since as early as the 70s.
Recently, the field of SR has seen major advances because of the rise of computing power (GPUs) which allowed innovation in machine learning and artificial intelligence algorithms.
Now we have access to voice control through speech recognition in mobile devices, computers, smart TVs or even fridges.

Before the emergence of deep learning, researchers often utilized other classification algorithms such as Hidden Markov Model (HMM) with many complex handcrafted components.
The field is now gradually moving towards end-to-end speech recognizer using just a neural networks which learns to transcribe an audio sequence signal directly to a word sequence, one character at a time.
Therefore, all the handcrafted components would be replaced with a just one learning model.


In this thesis, we present the concept of artificial neural networks (ANN), basics of the internal network architecture and explained the training phase of ANN.
We extend the knowledge of neural networks by introducing recurrent neural networks and most importantly we cover how speech recognition system works and how can we build end-to-end SR using neural networks.

Our goal is to get theoretical overview in this field and implement end-to-end speech recognizer using neural networks and TensorFlow library which would be used in Robot NAO as voice-user interface on Robot NAO.

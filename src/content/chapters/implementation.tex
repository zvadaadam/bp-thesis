\chapter{Implementation}

The goal is to implement end-to-end speech recognizer using neural network.
High-level concept, how the implemented speech recognition system works is illustred on *figure*.
It takes a wav file as an input generated from given microphone and perfroms preprocessing and feature extraction.
The data are feeded to the recognizer which outputs the prediction of transcribed text from speech.

\imagefigurelarge{prediction_diagram.pdf}{Speech Recognition System}

In order to perfrom sucessful prediction with the recognizer, it needs to be trainned on tratning data.

\imagefigurelarge{learning_diagram.pdf}{Diagram of the learning phase for the speech recognition system}

\section{Tools}

\subsection{Python}

Speech recognition system is implemented in programming languge \textit{Python} which is currently most popular approach in machine learning and AI.
Python is a very powerful, flexible, open source language that is easy to learn.
The greastest strenth however is wide range of libraries and frameworks for ML and AI.

\subsection{Tensorflow}


TensorFlow is open-source library developed by Google for deep learning and other algorithms involving large number of mathematical operations.
The primary unit in TensorFlow is a tensor.
A tensor consists of a set of primitive values shaped into an array of any number of dimensions.
These massive numbers of large arrays are the reason that GPUs and other processors designed to do floating point mathematics excel at speeding up these algorithms. *Reference*
% https://dzone.com/refcardz/introduction-to-tensorflow?chapter=1

TensorFlow programs are structured into a construction phase that assembles a comuptaional graph, and an execution phase that uses a session to execute operation in the graph.
However, TensorFlow programs are hard to debug because of the structure. Fortunately, TensorFlow offers a built-in function for visualization of the computaion called TensorBoard.

\section{Training Data}

Since we implement end-to-end speech recognizer

\section{Computing Power}

\section{Preprocessing and Feature Extraction}

\section{Recognizer}
